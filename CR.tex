\documentclass[4apaper]{report}
\RequirePackage{amsmath}

\author{David Bozon, Franck Muraton}

\begin{document}
\title{Report for TP1 Cryptography}
\maketitle

\paragraph{Instructions: \\} This is the report of the lab work for the cryptography class. To use the project, compile using \texttt{make}, and launch with an argument using '\texttt{./tp1}'. The available arguments are '\texttt{--part1}' which use the functions of part 1, except question 5; '\texttt{--question5}' which does the question 5 (which takes a while to be done, less than a minute), '\texttt{--part2}', which use the functions of part 2, except for question 9 and 11 which take a while to be done. Each of those function can be called individually (using '\texttt{--question9}' or '\texttt{--question11}') For exemple: \texttt{make \&\& ./tp1 --question9}. \\ Assertions are used when test values are provided to make sure we have the correct values. \\Any question can be called using the flag '\texttt{--questionN}', as long as $N \in \{1,3,4,5,6,7,9,10,11\}$. \\To call every questions, you can use '\texttt{./tp1 --all}'.

\paragraph{Question 2 : Do you think that the key size of \texttt{sip\_hash\_2\_4} is large enough to make an exhaustive search of the keyspace intractable? Can this function be considered to be collision-resistant?}

\paragraph{}Here, we have a $2^{128}$ key space. According to the first lesson of the class, if we want to be sure to have a result whitin 34 years ($2^{30}$ seconds), we'd need hardware with $2^{50}$ iterations per second, this should be trivially parallelizable, and each device should use 1000W without overhead.

Which, still according to the lesson, mean that we need $2^{128-50-30}$, so $2^{48}$ machines, which is approximately 280 000 000 GW, about the output of 170 000 000 EPR nuclear power plants.

But for the collision, we have a problem. The Birthday paradox mean that we have a generic complexity of $2^{n/2}$, so here, we have 4294967296 possibilities, which is not that big.

\paragraph{Question 4 : When writing your report, be particularly careful when explaining your implementation
choices for this question.}

\paragraph{} At first we tried a naive implementation : compute the hash every time, and compare them using a loop. Needless to say, as we are writing the report the loop is still running. This is not an efficient way to do.

We also tried pre-computing a hash table for a given key. The computing wasn't over and the table was over 20GB in size, and this was for ONE key.

We then decided to try a smarter way of finding a collision, using the birthday paradox. The probability that one message has the same digest as another is of $\frac{1}{1/(2^{32} - 1)}$. But as we add more message, we can start improving our chances of finding a collision. This mean that we will not have to compute every possible hash for a decent chance of finding a collision. If we compute $2^{32}-1$ messages, our chances are of $100\%$, according to the pigeon-hole principle (if you have $n$ cages, and $n+1$ pigeon, if you put a pigeon in every empty cage then you will need to have a cage with two pigeons in it).

The probability of someone sharing a birthday with someone else in a room of $n$ people is of $1 - (\frac {364}{365})^{\binom {n}{2}}$. Since we have $2^{32}-1$ possibilities here, the corresponding formula for our hash would be  $1 - (\frac {2^{32}-2}{2^{32}-1})^{\binom {n}{2}}$. For $2^{18}$ hash, the probability of having a collision is of $0.99966\%$ and with $2^{19}$, we have probability of $1$. This mean that we only need to compute $2^{19}$ hash to find a collision, which is much better than have to compute $2^{32}$ hashes.

We decided to use a C++ \texttt{std::unordered\_map}, which let us insert only unique keys, which are our hashes. If we happen to insert a key that already exist, we can know about it from the object being returned. So we just have to start building the table, and check for an insertion failure, which will indicate a collision. No extra search is required, and the rainbow table might not even have to be fully computed.

\paragraph{Question 5 : Gather statistics about the expected time (as a number of function calls) necessary to get a
collision for \texttt{sip\_hash\_fix32} "in counter mode" for a fixed key. In particular, give the average, minimum, and maximum time over 1000 (or more) distinct keys. Are these results consistant
with your initial estimate?}

\paragraph{} $2^{19}$ is the maximum function call that need to be done to have a guarentee of finding a collision. On average we need about 25 ms to find a collision, the maximum time required is about 112.022 ms and the minimum time required is about 0.182 ms, on a Intel(R) Core(TM) i5-3317U.

\paragraph{Question 8: Do you think that the key size of twine\_fun1 is large enough to make an exhaustive search of
the keyspace intractable? Can this function be considered to be collision-resistant?}

\paragraph{} Here, we have $2^{32}$ key possibilities, which is far less than what we had for \texttt{sip\_hash\_fix32}. An exhaustive search is indeed possible, as computing $2^{32}$ \texttt{twine\_fun1} on the same machine as question 5 takes less than 5 seconds. The complexity of computing a collision is of $2^{n/2}$, which mean that it is trivial in our case ($2^{16}$). This function can't be considered to be collision-resistant.

\paragraph{Question 9: Gather statistics about the expected time (as a number of function calls) necessary to get a
collision for \texttt{twine\_fun1} “in counter mode” for a fixed key. In particular, give the average, minimum, and maximum time over 1000 (or more) distinct keys. How do these results compare with \texttt{sip\_hash\_fix32} ?}

\paragraph{} $2^{19}$ is the maximum function call that need to be done to have a guarentee of finding a collision. On average we need about 34 ms to find a collision, the maximum time required is about 63 ms and the minimum time required is about 31 ms, on a Intel(R) Core(TM) i5-3317U. The minimum time required is higher than with \texttt{sip\_hash\_fix32}, however the maximum time required is also lower. We end up with an average time just above.

\paragraph{Question 11: Gather statistics about the expected time (as a number of function calls) necessary to get a
collision for \texttt{twine\_fun2\_fix32} and \texttt{twine\_fun2\_fix16} “in counter mode” for a fixed key. In particular, give the average, minimum, and maximum time over 1000 (or more) distinct keys. How do these results compare with \texttt{sip\_hash\_fix32} and \texttt{twine\_fun1} ? How can you explain these results?}

\end{document}
